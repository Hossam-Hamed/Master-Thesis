\chapter{Background}
\label{ch:background}
This chapter gives an overview over concepts and related required for understanding the protocols and algorithms discussed in further chapters of the thesis.
\section{Preliminaries}

\subsection{Cryptography Concepts Overview}
The practice and study of mechanisms for secure communication is known as cryptography. Secure communication in digital communications is the assurance of message privacy between communicating parties, even when the communication channel is untrusted, and adversaries are present. Cryptography is the process of creating and evaluating protocols with the prime objective of assuring data confidentiality, integrity, and authenticity between communicating parties. The fundamental basis of cryptography are mathematical theory and computer science practice. Cryptographic algorithms are built around assumptions about computational difficulty. While it is theoretically conceivable to break such algorithms in a well-designed system, any adversary would find it infeasible in practice.
\par
Encryption is the operation of converting information (\textit{cleartext}) using cryptographic algorithms into a form (\textit{ciphertext}) that is unintelligible to the public or an adversary. The ciphertext can be restored to its original form by its intended recipient who owns the necessary cryptographic material to perform the decryption process. Encryption is a prominent and commonly utilized method to ensure confidentiality.
\par
On the other hand, integrity is the assurance that data has not been intercepted and modified by an adversary or corrupted in transit. Mainly, the approach to ensure data integrity is through mathematically compute a digital fingerprint for the data via hash functions. Subsequently, the digital fingerprint is attached to the data and sent to their designated party. The receiver verifies that the fingerprint corresponds to the data received, and thus the data integrity is proven.
\par
Cryptography is split into two main cryptosystems: Symmetric and Asymmetric cryptography.

\subsubsection{Symmetric Cryptography}
Symmetric cryptography, aka shared secret cryptography, relies in its operations on a private key that is shared between the communicating parties. In symmetric encryption, the shared secret is used for both encryption and decryption operations. Schemes for symmetric encryption are divided into two main categories according to how data is encrypted: Block ciphers and Stream ciphers.
Block ciphers encrypt a single fixed-size block of data at a time. Because block ciphers utilize deterministic algorithms, a plaintext block will always encrypt to the same ciphertext when the same key is used. Block ciphers can operate in several different modes, including CBC and CTR. The AES algorithm is an example of a well-known secure block cipher.
A stream cipher is a symmetric key cipher in which each cleartext bit is XORed with a corresponding keystream bit one bit at a time. The output stream is generated based on an internal state that is hidden and changes while the encryption functions. An arbitrarily lengthy stream of secret key material is used to establish that internal state at first. The key is constantly changing by implementing a form of feedback mechanism. Stream ciphers, in general, operate faster than block ciphers and have less hardware complexity. RC4 is an example for a widely used stream cipher.
\par
Integrity, as well as authenticity, in symmetric cryptography are achieved through \glspl{mac}. \gls{mac} algorithms are realized through key-based hash functions, e.g. HMAC. Therefore, \gls{mac} algorithms inherit several properties from hash functions. For example, They accept messages of arbitrary size and output a \gls{mac} of a fixed size, relative to the algorithm used. A \gls{mac} is difficult to forge or to find a collision for. Moreover, a \gls{mac} is permanently tied to the specific message it is created for.
The \gls{mac} is sent along with the message it verifies to the other party, which in turn recomputes the \gls{mac} of the received message using the shared secret and compares the output of its algorithm to the \gls{mac} it received. If both \glspl{mac} match, then the message is authentic and tamper-free. Otherwise, the message should be discarded.

\subsubsection{Asymmetric Cryptography}
Asymmetric cryptography, aka public key cryptography, is established upon the use of two different keys, a public key and a private key. Although the keys are different, they are mathematically related. Nevertheless, the key pair are constructed in a fashion that it is infeasible to compute one from the other. The key pair generation relies on computational complexity of hard mathematical functions, such as the discrete logarithm problem. The most common practice is that a party creates its own key pair. The public key is published to public peers; however, the private key remains a secret to its owner. In an Alice and Bob notation, if Alice wants to encrypt a message to Bob, Alice uses Bob's public key for encryption. Bob can decrypt the received message using his private key. A famouse example for a public key encryption algorithm is RSA. Moreover, the key pair can be used in a key establishment protocol like \gls{dh}.
\par
To guarantee message integrity and authenticity, digital signatures are used to proof that a message is tamper-free. Digital signatures are analogous to \glspl{mac} in symmetric cryptography. In contrast to encryption, when Alice wants to sign a message that will be sent to Bob, Alice uses its private key in the signature algorithm to produce a signature. The signature is attached to the message and sent to Bob. Bob uses Alice's public key to verify the signature of the received message. Two of the most used digital signature methods are RSA and DSA.

\subsection{Digital Certificates}
In the realm of public key cryptography, an \gls{ee} in possession of a key pair is able to encrypt, decrypt and sign data to and from other entities. This satisfies the confidentiality and integrity requirements of communication security. However, this is not sufficient to achieve the authentication and authorization goals
certificates, which are a requirement in several use cases. Authentication is the assurance that an entity is in fact who it claims to be, while authorization is the assurance that an entity has the privilege to communicate with a resource. Without authentication guarantees, a malicious third party can impersonate any other party in a communication channel or manipulate the messages between two communicating parties. Moreover, the intrusion cannot be detected. To address this problem, digital certificates were introduced.
\par
A digital certificate (aka public key certificate) is an electronic document that cryptographically binds a public key to the identity of an \gls{ee}. The certificate is cryptographically signed by a trusted third party that issues the certificate. In addition to including the public key and the digital signature, the certificate also contains information about the public key, the identity of its owner, and information about the certificate issuer. The party that wishes to authenticate itself presents its digital certificate to its counterpart. To validate a certificate, the receiving party has to trust the certificate issuer. If the issuer is trusted, then its public key is used to verify the certificate's signature and as a result trust the presented certificate.
\par
To be able to use digital certificates in a scalable environment like the internet, a trusted large scale infrastructure that allows for the generation, storage, and distribution and revocation of certificates is required.
Although not the only one, the X.509 standard \cite{x509} is the most commonly used for specifying digital certificates format and \gls{pki}. It is an product of International Telecommunication Union (ITU). X.509 certificates are widely used in many internet protocols, including TLS. Therefore, X.509 is considered a pillar of network security. Currently, there are three versions of X.509, the most recent of which is simply known as X.509 version 3. In addition to the certificate identifiers and attributes, optional extensions have been added to the newest version, which can be tagged as critical and thus must be handled by the recipient, or can be ignored otherwise. In the X.509 \gls{pki}, a \gls{ca} is an entity responsible for handling end entities request to generate a certificate for their public keys from the \gls{pki}. The requests are known as \gls{csr}. The \gls{ca} verifies the identity of the \gls{ee} and checks the attributes of the \gls{csr} subsequently decides whether to issue and sign a certificate to the requester or not. In the X.509 PKI, the \gls{ca} is a trusted third party and the issued certificate is dubbed an \textit{\gls{ee} certificate}. Nevertheless, the \gls{ca} needs to have a certificate to identify itself. In a \gls{pki} there are several \glspl{ca} where they are organized in a hierarchical architecture. At the top level, there are the \textit{Root \gls{ca}} which have self-signed certificates as they are the highest authority. Root \glspl{ca} issue certificates to \glspl{ca} in the same level (cross certification) or in lower levels. Lower level \glspl{ca} are named \textit{intermediate \glspl{ca}} which can issue certificates to end entities. Though end entities cannot use their certificates to issue other certificates. Another component of a PKI is the \gls{ra}. A \gls{ca} can delegate some of its functionality to a \gls{ra}. A \gls{ra} lies between the \gls{ca} and the \gls{ee}, in addition, it is usually in proximity to the \gls{ee}.
\par
\gls{pki} provide methods for certificate revocation. Naturally, a certificate is only valid for a limited time due to various constrains. There are numerous critical reasons why its validity must be terminated sooner than allotted due to various threats, and hence the certificate must be revoked. For example, compromise of \gls{ee} or a trusted \gls{ca} private key. Wohlmacher \cite{revocationSurvey} provides a survey study of revocation methods that gives a good overview of the main revocation methods.

\subsection{Bootstrapping}
The term ``Bootstrapping" is used in a spectrum of contexts including Computing, Law, Finance\footfullcite{bs-wiki}. It is inspired from the late idiom ``to pull oneself up by one's bootstraps" which means to succeed or elevate yourself without any outside help. In the context of Networking, bootstrapping is an initial procedure between an unconfigured devices which intend to communicate with a network for the first time. Its goal is to provide the device with the required information that enables it to establish subsequent secure communication channels with the desired network. Such information can be certificates, configurations, and metadata. More on bootstrapping in chapter \ref{ch:secureBootstrapping}.


\subsection{Voucher Artifact}
While bootstrapping, the pledge must be able to authenticate the network attempting to take control of it, hence assigning ownership is critical for bootstrapping techniques. Defined in \cite{rfc8366}, a voucher is a signed document that identifies the cryptographic identity of the domain that a pledge should trust. The goal of the voucher is to allow for a authorize a zero-touch imprinting of the pledge on the registrar of a domain. The voucher is signed by a manufacturer's \gls{masa}. The voucher artifact is a JSON formatted document which is included and signed in a CMS structure \cite{rfc5652}. The artifact data model is formally described by a YANG \cite{rfc7950} module in \cite[Section~5.3]{rfc8366}. Essentially, the main purpose of a voucher is to securely convey a certificate to the pledge with which it can authenticate the owner's domain registrar. The said certificate is found in the ``pinned-domain-cert" attribute of the voucher. 
\par
Vouchers can be classified into two types: nonced and nonceless vouchers. Nonced vouchers contain the same nonce specified by the pledge in its submitted voucher request. Nonces provide mitigation against replay attacks as they are not reusable. On the other hand, Nonceless vouchers may be reusable. Their validity is dependent on their lifetime which may vary. A voucher's lifetime is specified by its ``expires-on" attribute. 
Since voucher are non-revocable artifacts, the specification recommends short-lived vouchers rather than long-lived vouchers with the possibility to renew a voucher by reissuing it. The specification points out that the renewal process should be a lightweight process, as it ostensibly only updates the voucher's validity period. There may also exist nonceless vouchers without expiration times, however, they are advised against as they provide their bearer with extreme control over the pledge. Lastly, some pledges may not have the capability of understanding time, such pledges must not use vouchers with time constrains.

\subsection{Certificate Enrollment}
Certificate enrollment is the process where an \gls{ee} requests and obtains a digital certificate from a \gls{pki}. The process starts by the \gls{ee} submitting a \gls{csr} to the \gls{ca} or \gls{ra} of the \gls{pki} which the \gls{ee} intends to obtain a digital certificate from. Several protocols \cite{rfc7030,rfc4210,rfc8894,rfc5272} were proposed to address certificate enrollment.
\par
For certificate enrollment protocols, it is mandatory for \gls{ee} to authenticate messages from \glossary{pki} entities. This is achieved by installing a root CA certificate on the \gls{ee} that enables it to authenticate the PKI entities. Nevertheless, provisioning such certificate to an EE is out-of-band of certificate enrollment protocols, but it is achievable through bootstrapping protocols. On the other hand, the \gls{pki} may authenticate \glspl{ee} using certificates they already posses or through private shared secrets.
\par
Some protocols, like \cite{rfc8894} require the EE to generate their own key pair locally, but some protocols, like \cite{rfc7030,rfc4210}, provide the possibility to generate the key pair for the EE during certification and transfer the keys to the EE during the process. In either case, the EE has to provide proof of possession of the private key, possibly through digital signature, decryption, or challenge-response.
Messages in certificate enrollment protocols have to be protected either by digital signatures or MACs.
\par
Certificate enrollment protocols generally realize a set of basic functionalities. These functionalities include certificate enrollment, certificate update and renewal, and certificate revocation. However, how each protocol implements a functionality can differ in its features. For instance, \cite{rfc5272} allows \glspl{ee} and/or \glspl{ra} to revoke \glspl{ee} certificates, while \cite{rfc8894} allows only \glspl{ca} to revoke certificates.
Different certificate enrollment protocols offer different functionalities. Some are only concerned with communication between \glspl{ee} and CA/RA, while others offer intra-PKI entities communications. For example, \cite{rfc4210} realizes cross-certification between \glspl{ca}.

\subsection{Protocol Formal Verification}\label{bg:pfm}
Formal verification is the application of mathematical procedures to confirm that a design complies with some clearly specified notion of functional correctness. In the lack of formal mechanisms for verification, security flaws may go unnoticed. On the other hand, formal verification approaches give a method for analyzing complex protocol in detail and prove the absence of large classes of attacks in a systematic fashion. Automated verification tools are programs built around a specific approach of formal methods with the goal of providing security guarantees for protocol execution employing the underlying cryptographic primitives, cryptographic protocols, and network systems. They help uncover protocol situations or vulnerabilities that contradict the intuitive assumptions underlying the construction of a protocol since they can analyze a large number of scenarios based on rigorous mathematical notions.
\par
Symbolic model checking is a protocol verification technique for determining if a protocol's finite-state model fits a set of requirements. Symbolic protocol analyzers effectively identify attacks, but they provide poorer security assurances than standard cryptographic proofs that account for probabilistic and computational concerns since they treat cryptographic constructions as perfect black boxes. \gls{ofmc} \cite{ofmc}  is a symbolic analyzer for security protocols. The AVISPA Intermediate Format IF \cite{avispa} is OFMC's native input language. AnB \cite{AnB} is an intuitive Alice-and-Bob-style language that OFMC also supports. OFMC automatically converts AnB to IF. It performs protocol falsification and bounded session verification by exploring the transition system resulting from an IF specification. OFMC successfully employs two primary techniques: the lazy invader and constraint differentiation. The lazy invader is a symbolic representation of the Dolev-Yao intruder \cite{dolev1983security} that functions in a demand-driven manner. Constraint differentiation is a broad search-reduction approach that combines the lazy intruder with partial-order reduction principles.

\subsection{Key Derivation Function (KDF)} \label{backgroung:kdf}
A \gls{kdf} is a cryptographic algorithm that generates keying material that cryptographic algorithms can use. The function requires two sorts of input: a secret value, such as a key or a password, and other data. The core of a KDF is often built using a pseudorandom function, such as Keyed cryptographic hash functions. A key derivation function iterates an n-bit pseudorandom function and concatenates the outputs until L bits of keying material are generated. Each output receives a distinct value with each cycle, thus if one key is compromised, the risk is isolated to that key while preceding keys remain secure. \gls{nist} proposed recommended techniques to use pseudorandom-based \glspl{kdf} in \cite{chen2008recommendation}. 
\begin{figure}[hptb]
	\centering
	\includegraphics[scale=0.65]{Images/kdf.png}
	\caption{Key derivation function.}
	\label{fig:kdf}
\end{figure}

\subsection{Threat Models}

\subsection{Secure Messaging}
	
	\subsubsection{properties}
	%	
	%	\item Forward Secrecy:
	%	
	%	\item Future Secrecy:
	%		- abstract ... explained in post-comp chapter
	%	\item immediate decryption:
	%		the fact that messages might arrive out of order or be lost entirely. Additionally, parties can be offline for extended periods of time and send and receive messages asynchronously. Given these inherent constraints, immediate decryption is a very attractive feature. Informally, it ensures that when a legitimate message is (eventually) delivered to the recipient, the recipient can not only immediately decrypt the message but is also able to place it in the correct spot in relation to the other messages already received. Furthermore, immediate decryption also ensures an even 	more critical liveness property, termed message-loss resilience (MLR) in this work: if a message is permanently lost by the network, parties should still be able to communicate
	
	
\section{Related Work}
	\subsection{OTR}
\subsection{SoK}
\subsection{Key Continuity Management}
\subsection{On post-compromise security}
study post-compromise security 	in (classic) key exchange. Here, security shall be achieved even for sessions established after a full compromise of user secrets. This necessarily requires mixing user state information with key material that is newly established via asymmetric techniques, and is thus related to RKE.
\subsection{Tesla}
