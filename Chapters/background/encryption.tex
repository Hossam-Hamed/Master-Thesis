The practice and study of mechanisms for secure communication is known as cryptography. Secure communication in digital communications is the assurance of message privacy between communicating parties, even when the communication channel is untrusted, and adversaries are present. Cryptography is the process of creating and evaluating protocols with the prime objective of assuring data confidentiality, integrity, and authenticity between communicating parties. The fundamental basis of cryptography are mathematical theory and computer science practice. Cryptographic algorithms are built around assumptions about computational difficulty. While it is theoretically conceivable to break such algorithms in a well-designed system, any adversary would find it infeasible in practice.
\par
Encryption is the operation of converting information (\textit{cleartext}) using cryptographic algorithms into a form (\textit{ciphertext}) that is unintelligible to the public or an adversary. The ciphertext can be restored to its original form by its intended recipient who owns the necessary cryptographic material to perform the decryption process. Encryption is a prominent and commonly utilized method to ensure confidentiality.
\par
On the other hand, integrity is the assurance that data has not been intercepted and modified by an adversary or corrupted in transit. Mainly, the approach to ensure data integrity is through mathematically compute a digital fingerprint for the data via hash functions. Subsequently, the digital fingerprint is attached to the data and sent to their designated party. The receiver verifies that the fingerprint corresponds to the data received, and thus the data integrity is proven.
\par
Cryptography is split into two main cryptosystems: Symmetric and Asymmetric cryptography.

\subsubsection{Symmetric Cryptography}
Symmetric cryptography, aka shared secret cryptography, relies in its operations on a private key that is shared between the communicating parties. In symmetric encryption, the shared secret is used for both encryption and decryption operations. Schemes for symmetric encryption are divided into two main categories according to how data is encrypted: Block ciphers and Stream ciphers.
Block ciphers encrypt a single fixed-size block of data at a time. Because block ciphers utilize deterministic algorithms, a plaintext block will always encrypt to the same ciphertext when the same key is used. Block ciphers can operate in several different modes, including CBC and CTR. The AES algorithm is an example of a well-known secure block cipher.
A stream cipher is a symmetric key cipher in which each cleartext bit is XORed with a corresponding keystream bit one bit at a time. The output stream is generated based on an internal state that is hidden and changes while the encryption functions. An arbitrarily lengthy stream of secret key material is used to establish that internal state at first. The key is constantly changing by implementing a form of feedback mechanism. Stream ciphers, in general, operate faster than block ciphers and have less hardware complexity. RC4 is an example for a widely used stream cipher.
\par
Integrity, as well as authenticity, in symmetric cryptography are achieved through \glspl{mac}. \gls{mac} algorithms are realized through key-based hash functions, e.g. HMAC. Therefore, \gls{mac} algorithms inherit several properties from hash functions. For example, They accept messages of arbitrary size and output a \gls{mac} of a fixed size, relative to the algorithm used. A \gls{mac} is difficult to forge or to find a collision for. Moreover, a \gls{mac} is permanently tied to the specific message it is created for.
The \gls{mac} is sent along with the message it verifies to the other party, which in turn recomputes the \gls{mac} of the received message using the shared secret and compares the output of its algorithm to the \gls{mac} it received. If both \glspl{mac} match, then the message is authentic and tamper-free. Otherwise, the message should be discarded.

\subsubsection{Asymmetric Cryptography}
Asymmetric cryptography, aka public key cryptography, is established upon the use of two different keys, a public key and a private key. Although the keys are different, they are mathematically related. Nevertheless, the key pair are constructed in a fashion that it is infeasible to compute one from the other. The key pair generation relies on computational complexity of hard mathematical functions, such as the discrete logarithm problem. The most common practice is that a party creates its own key pair. The public key is published to public peers; however, the private key remains a secret to its owner. In an Alice and Bob notation, if Alice wants to encrypt a message to Bob, Alice uses Bob's public key for encryption. Bob can decrypt the received message using his private key. A famouse example for a public key encryption algorithm is RSA. Moreover, the key pair can be used in a key establishment protocol like \gls{dh}.
\par
To guarantee message integrity and authenticity, digital signatures are used to proof that a message is tamper-free. Digital signatures are analogous to \glspl{mac} in symmetric cryptography. In contrast to encryption, when Alice wants to sign a message that will be sent to Bob, Alice uses its private key in the signature algorithm to produce a signature. The signature is attached to the message and sent to Bob. Bob uses Alice's public key to verify the signature of the received message. Two of the most used digital signature methods are RSA and DSA.