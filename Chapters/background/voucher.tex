While bootstrapping, the pledge must be able to authenticate the network attempting to take control of it, hence assigning ownership is critical for bootstrapping techniques. Defined in \cite{rfc8366}, a voucher is a signed document that identifies the cryptographic identity of the domain that a pledge should trust. The goal of the voucher is to allow for a authorize a zero-touch imprinting of the pledge on the registrar of a domain. The voucher is signed by a manufacturer's \gls{masa}. The voucher artifact is a JSON formatted document which is included and signed in a CMS structure \cite{rfc5652}. The artifact data model is formally described by a YANG \cite{rfc7950} module in \cite[Section~5.3]{rfc8366}. Essentially, the main purpose of a voucher is to securely convey a certificate to the pledge with which it can authenticate the owner's domain registrar. The said certificate is found in the ``pinned-domain-cert" attribute of the voucher. 
\par
Vouchers can be classified into two types: nonced and nonceless vouchers. Nonced vouchers contain the same nonce specified by the pledge in its submitted voucher request. Nonces provide mitigation against replay attacks as they are not reusable. On the other hand, Nonceless vouchers may be reusable. Their validity is dependent on their lifetime which may vary. A voucher's lifetime is specified by its ``expires-on" attribute. 
Since voucher are non-revocable artifacts, the specification recommends short-lived vouchers rather than long-lived vouchers with the possibility to renew a voucher by reissuing it. The specification points out that the renewal process should be a lightweight process, as it ostensibly only updates the voucher's validity period. There may also exist nonceless vouchers without expiration times, however, they are advised against as they provide their bearer with extreme control over the pledge. Lastly, some pledges may not have the capability of understanding time, such pledges must not use vouchers with time constrains.