Formal verification is the application of mathematical procedures to confirm that a design complies with some clearly specified notion of functional correctness. In the lack of formal mechanisms for verification, security flaws may go unnoticed. On the other hand, formal verification approaches give a method for analyzing complex protocol in detail and prove the absence of large classes of attacks in a systematic fashion. Automated verification tools are programs built around a specific approach of formal methods with the goal of providing security guarantees for protocol execution employing the underlying cryptographic primitives, cryptographic protocols, and network systems. They help uncover protocol situations or vulnerabilities that contradict the intuitive assumptions underlying the construction of a protocol since they can analyze a large number of scenarios based on rigorous mathematical notions.
\par
Symbolic model checking is a protocol verification technique for determining if a protocol's finite-state model fits a set of requirements. Symbolic protocol analyzers effectively identify attacks, but they provide poorer security assurances than standard cryptographic proofs that account for probabilistic and computational concerns since they treat cryptographic constructions as perfect black boxes. \gls{ofmc} \cite{ofmc}  is a symbolic analyzer for security protocols. The AVISPA Intermediate Format IF \cite{avispa} is OFMC's native input language. AnB \cite{AnB} is an intuitive Alice-and-Bob-style language that OFMC also supports. OFMC automatically converts AnB to IF. It performs protocol falsification and bounded session verification by exploring the transition system resulting from an IF specification. OFMC successfully employs two primary techniques: the lazy invader and constraint differentiation. The lazy invader is a symbolic representation of the Dolev-Yao intruder \cite{dolev1983security} that functions in a demand-driven manner. Constraint differentiation is a broad search-reduction approach that combines the lazy intruder with partial-order reduction principles.