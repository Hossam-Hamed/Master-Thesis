In the realm of public key cryptography, an \gls{ee} in possession of a key pair is able to encrypt, decrypt and sign data to and from other entities. This satisfies the confidentiality and integrity requirements of communication security. However, this is not sufficient to achieve the authentication and authorization goals
certificates, which are a requirement in several use cases. Authentication is the assurance that an entity is in fact who it claims to be, while authorization is the assurance that an entity has the privilege to communicate with a resource. Without authentication guarantees, a malicious third party can impersonate any other party in a communication channel or manipulate the messages between two communicating parties. Moreover, the intrusion cannot be detected. To address this problem, digital certificates were introduced.
\par
A digital certificate (aka public key certificate) is an electronic document that cryptographically binds a public key to the identity of an \gls{ee}. The certificate is cryptographically signed by a trusted third party that issues the certificate. In addition to including the public key and the digital signature, the certificate also contains information about the public key, the identity of its owner, and information about the certificate issuer. The party that wishes to authenticate itself presents its digital certificate to its counterpart. To validate a certificate, the receiving party has to trust the certificate issuer. If the issuer is trusted, then its public key is used to verify the certificate's signature and as a result trust the presented certificate.
\par
To be able to use digital certificates in a scalable environment like the internet, a trusted large scale infrastructure that allows for the generation, storage, and distribution and revocation of certificates is required.
Although not the only one, the X.509 standard \cite{x509} is the most commonly used for specifying digital certificates format and \gls{pki}. It is an product of International Telecommunication Union (ITU). X.509 certificates are widely used in many internet protocols, including TLS. Therefore, X.509 is considered a pillar of network security. Currently, there are three versions of X.509, the most recent of which is simply known as X.509 version 3. In addition to the certificate identifiers and attributes, optional extensions have been added to the newest version, which can be tagged as critical and thus must be handled by the recipient, or can be ignored otherwise. In the X.509 \gls{pki}, a \gls{ca} is an entity responsible for handling end entities request to generate a certificate for their public keys from the \gls{pki}. The requests are known as \gls{csr}. The \gls{ca} verifies the identity of the \gls{ee} and checks the attributes of the \gls{csr} subsequently decides whether to issue and sign a certificate to the requester or not. In the X.509 PKI, the \gls{ca} is a trusted third party and the issued certificate is dubbed an \textit{\gls{ee} certificate}. Nevertheless, the \gls{ca} needs to have a certificate to identify itself. In a \gls{pki} there are several \glspl{ca} where they are organized in a hierarchical architecture. At the top level, there are the \textit{Root \gls{ca}} which have self-signed certificates as they are the highest authority. Root \glspl{ca} issue certificates to \glspl{ca} in the same level (cross certification) or in lower levels. Lower level \glspl{ca} are named \textit{intermediate \glspl{ca}} which can issue certificates to end entities. Though end entities cannot use their certificates to issue other certificates. Another component of a PKI is the \gls{ra}. A \gls{ca} can delegate some of its functionality to a \gls{ra}. A \gls{ra} lies between the \gls{ca} and the \gls{ee}, in addition, it is usually in proximity to the \gls{ee}.
\par
\gls{pki} provide methods for certificate revocation. Naturally, a certificate is only valid for a limited time due to various constrains. There are numerous critical reasons why its validity must be terminated sooner than allotted due to various threats, and hence the certificate must be revoked. For example, compromise of \gls{ee} or a trusted \gls{ca} private key. Wohlmacher \cite{revocationSurvey} provides a survey study of revocation methods that gives a good overview of the main revocation methods.