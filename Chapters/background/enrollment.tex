Certificate enrollment is the process where an \gls{ee} requests and obtains a digital certificate from a \gls{pki}. The process starts by the \gls{ee} submitting a \gls{csr} to the \gls{ca} or \gls{ra} of the \gls{pki} which the \gls{ee} intends to obtain a digital certificate from. Several protocols \cite{rfc7030,rfc4210,rfc8894,rfc5272} were proposed to address certificate enrollment.
\par
For certificate enrollment protocols, it is mandatory for \gls{ee} to authenticate messages from \glossary{pki} entities. This is achieved by installing a root CA certificate on the \gls{ee} that enables it to authenticate the PKI entities. Nevertheless, provisioning such certificate to an EE is out-of-band of certificate enrollment protocols, but it is achievable through bootstrapping protocols. On the other hand, the \gls{pki} may authenticate \glspl{ee} using certificates they already posses or through private shared secrets.
\par
Some protocols, like \cite{rfc8894} require the EE to generate their own key pair locally, but some protocols, like \cite{rfc7030,rfc4210}, provide the possibility to generate the key pair for the EE during certification and transfer the keys to the EE during the process. In either case, the EE has to provide proof of possession of the private key, possibly through digital signature, decryption, or challenge-response.
Messages in certificate enrollment protocols have to be protected either by digital signatures or MACs.
\par
Certificate enrollment protocols generally realize a set of basic functionalities. These functionalities include certificate enrollment, certificate update and renewal, and certificate revocation. However, how each protocol implements a functionality can differ in its features. For instance, \cite{rfc5272} allows \glspl{ee} and/or \glspl{ra} to revoke \glspl{ee} certificates, while \cite{rfc8894} allows only \glspl{ca} to revoke certificates.
Different certificate enrollment protocols offer different functionalities. Some are only concerned with communication between \glspl{ee} and CA/RA, while others offer intra-PKI entities communications. For example, \cite{rfc4210} realizes cross-certification between \glspl{ca}.