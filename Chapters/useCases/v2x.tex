\gls{v2x} is the exchange of information between a vehicle and other entities that may impact or be impacted by the vehicle. It is a form of vehicular communication system that integrates other more particular types of communication:
\begin{itemize}
	\item Vehicle-to-Vehicle (V2V): It is the transmission of data between vehicles in the same proximity. Communication can be direct between two vehicles or relayed through another vehicle. 
	\item Vehicle-to-Infrastructure (V2I): A \textit{roadside unit (RSU)} is a device representing the infrastructure that exchanges data between a vehicle and the infrastructure.
	\item Vehicle-to-Network (V2N): It refers to the exchange of data between a vehicle and external networks such as could services or data centers.
	\item Vehicle-to-Pedestrian (V2P): It is the exchange of data between vehicles and pedestrians.
\end{itemize}
Many vehicular-related benefits can be realized through real-time communication of V2X, specifically V2V and V2I. On one side, governments are pressuring the industrial sector to accelerate the technology's development and deployment as it is envisioned to increase road safety and reduce fatalities and congestion through the potential use cases of the technology---for example, forward collision warning and emergency vehicle approaching warning. Similarly, car manufacturers rely on the technology to deploy practical fully autonomous vehicles and driving assistant solutions. 
\par
%External interfaces that enable wireless connectivity, known as onboard units (OBUs), are used for communication between the vehicle and the outside world, such as other vehicles or RSUs. To gather and transmit vehicular data, a vehicle's control unit collaborates with the OBU.
In a V2X setting, there are numerous messages for various applications that are exchanged between the entities of V2X communication. Despite the challenging design constraints to secure the communication due to the nature of the V2X use case, it is critical to provide security guarantees for this setting. The V2X communications requires a set of security goals to be achieved: confidentiality, authentication, and non-repudiation. Privacy is also a complex topic in this setting since authentication and non-repudiation rely on digital certificates and signature, which by default are not coherent with the privacy requirement. One suggestion for this issue is the use of anonymous certificates as proposed by \cite{8275624}.
ETSI provides a technical specification regarding intelligent transport systems security \cite{etsi_2021}. The specification classifies potential use cases into several classes. The behavior of each proposed use case communication pattern was further defined. The discussed use cases fall under the categories of V2V and V2I. It can be deduced that addressing methods for the use cases are divided between broadcast, multicast, and unicast messaging, with some of the multicast and unicast based-applications requiring the establishment of secure communication sessions. Moreover, for each group of applications, the specification discusses the security requirements in terms of authentication, authorization, confidentiality, and privacy. Finally, Hasan et al. \cite{9068410} present a survey that provides more in-depth information about V2X technologies and security/privacy standardization activities, existing architectures for securing V2X communications, and classification and summary of potential security threats for V2X applications.

