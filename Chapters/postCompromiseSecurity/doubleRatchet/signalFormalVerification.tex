\section{Formal Verification of Signal Protocol}
For completeness, we mention below former work related aimed at formally analyzing the Signal protocol without diving into details.
\par
Cohn-Gordon et al. \cite{cohn2020formal} were the first to address Signal's security in a formal manner. The verification methodology is highly comprehensive and sophisticated, and it is designed particularly for the Signal protocol \cite{alwen_coretti_dodis_2020}. A completely adversarially controlled network is used to evaluate the protocol. Through the definition of a security model, the research covers Signal's \gls{x3dh} and Double Ratchet protocols as a multi-stage authenticated key exchange protocol. The model depicts the ratcheting key update structure as a multi-stage model, where instead of just a sequence, it is a tree-like structure of stages that reflect the chains in Signal. The model enables different parties to run numerous, simultaneous sessions, each with its own set of stages. Secrecy and authentication of message keys in the computational model, under a rich compromise scenario, are the high-level features targeted to be verified by hand. Nevertheless, forward and future secrecy are implied goals, as derived session keys should be kept secret in a range of compromise circumstances. For instance, if a long-term secret is compromised but a medium or ephemeral secret is not, or if a state is compromised and a secure asymmetric stage happens afterwards. Since Signal does not cleanly separate key exchange from subsequent data messages, the model had to reorder some procedures to achieve this separation. In addition and contrary to Signal, the model does not re-use \gls{dh} keys for signatures. Finally, the research shows that Signal's cryptographic core delivers the desirable security attributes specified in the security model, based on normal cryptographic assumptions. Reassuringly, its design is free of any severe defects. The model, on the other hand, does not address Signal's instantaneous decryption feature, which is a distinctive feature that privileges it to other protocols that lack it \cite{alwen_coretti_dodis_2020}. Furthermore, because the model is exclusive to the Signal protocol, it cannot be used as a reference notion for \gls{rke} because it provides a lower degree of security than would be expected for \gls{rke} \cite{poettering2018asynchronous}.
\par
According to Alwen et al. \cite{alwen_coretti_dodis_2020}, one of the critical shortcomings of formal Signal-related investigations prior to their work \cite{poettering2018asynchronous,x1,x2,x3} is that they all accomplish FS and PCS by expressly forgoing immediate decryption and, therefore, message-loss resilience. Abandoning this property facilitates the algorithmic design for these provably secure Signal alternatives, but it also makes them inadequate in situations when message loss is a possibility. For instance, the protocol may employ an unstable transport technology like UDP, or a centralized server could be dropping messages due to a wide variety of unforeseen events. The authors contend that none of the studies available at the time were entirely adequate. To address this issue, they provided a formal and generic definition of secure communications that includes the necessary security properties: forward secrecy, future secrecy, and, in particular, immediate decryption. Their work is the first to address the analysis of the Signal protocol, primarily the double-ratchet aspect, without waiving the immediate decryption property and in a well-defined generalized model that is not only specific for the Signal protocol. The methodology of the paper is generalizing and abstracting out the reliance on the specific Diffie-Hellman key exchange by not relying on random oracles, in addition to clarifying the role of various cryptographic hash functions used inside the current Signal approach. The outcome was a generalized modular model of the Signal protocol which was developed using the following modules: continuous key agreement (CKA), forward-secure authenticated encryption with associated data (FS-AEAD), and a two-input hash function. The design of their model is inclusive of other well-studied forms of state compromise proposed by \cite{poettering2018asynchronous} and \cite{x2} without sacrificing the immediate decryption property.