\chapter{Secure Bootstrapping}
\label{ch:secureBootstrapping}

Integrity and confidentiality of information flow between the two ends of the communication are what defines end to end secure channels. Authentication, whether unilateral or mutual, is another fundamental aspect to achieve channel security. Existing protocols, such as \gls{tls}, can achieve End-to-end security between parties. However, analogous protocols rely digital certificates and credentials which are managed by local or third party \gls{pki}. Therefore to ensure correct and secure execution of protocols certificates must be securely provisioned to their corresponding identities. Typically at the manufacturing phase, the manufacturer usually installs globally unique manufacturer provided identifiers known as the \gls{idevid} \cite{5367679}. Its main use is for identity verification purposes and it should not be used to enforce data integrity nor confidentiality. Upon successful completion of the bootstrapping process, the device should posses identifiers that allows for subsequent establishment of secure channels with the network domain. An identifier in this set of identifiers is known as \gls{ldevid} \cite{5367679}.
\par
Bootstrapping approaches differ in the degree of manual user involvement and the amount of information on the device which must be pre-configured by the manufacturer. The survey by Sethi et al. \cite{irtf-t2trg-secure-bootstrapping-00} provides a classification for Bootstrapping mechanisms into general categories: Managed methods, \gls{p2p} and Ad-hoc methods, Opportunistic methods, and Hybrid methods.
\begin{itemize}
	\item \textit{Managed methods}: These bootstrapping approaches count on pre-established credentials and trust anchors for authentication and security. The required initial information and cryptographic material can be acquired either at manufacture time in the factory, or through \gls{oob} means, e.g. using a smart card or a USB. Examples of this method are \gls{eap-tls} \cite{eap-tls} and \gls{gba} \cite{gba}.
	
	\item \textit{\gls{p2p} methods}: Contrary to managed methods, these bootstrapping methods do not rely on any pre-established cryptographic material or information. Instead, the bootstrapping protocol results in credentials being established for subsequent secure communication. Typically, resulting credentials are authenticated using an \gls{oob} channel. This category of methods may be utilized if the manufacturer is incapable or untrustworthy to generate the desired credential.
	
	\item \textit{Opportunistic methods}: Unlike previous methods where the authenticity of the initially presented identity is verified, approaches which fall under this category rely on verification of the continuity of the initial identity provided. Bergmann et al. \cite{simplekeys} have developed a secure bootstrapping mechanism that is an example of this category. 
	
	\item \textit{Hybrid methods}: A wide range of deployed approaches use components from both managed and \gls{p2p} methods. Such approaches are categorized as hybrid methods.	
\end{itemize} 

This chapter discusses two voucher-based bootstrapping protocols that aim at being zero touch protocols by eliminating the user interference. These protocols are \gls{brski} \cite{brski} and \gls{sztp} \cite{sztp}. Both protocols fall under the managed methods category.

%brski section
\section{\acrfull*{brski}}
%intro
\gls{brski} is a product of the ANIMA working group of the \gls{ietf}. It is an automated bootstrapping protocol that enables an unconfigured device to discover and securely join an unfamiliar network domain it is installed in. \gls{brski} results in the device device acquiring an X.509 root certificate to authenticate the network domains' elements and establish consecutive secure channels. Moreover, the device can use the obtained certificate to perform further certificate enrollment protocols, like \gls{est} \cite{rfc7030}. \gls{brski} is capable of realizing a large scale of thousands of devices in a risk prone environment. For example, customer devices provided by ISPs which are directly shipped to the customers.

\subsection{Architecture Overview}

The environment of \gls{brski} is composed of three general entities: the network domain, the pledge, and the manufacturer services. Figure \ref{brski-architecture} shows an overview of the protocols architecture.

\begin{figure}[H]
	\centering
	\includegraphics[scale=0.4]{Images/brski-overview.png}
	\caption{BRSKI Architecture.}
	\label{brski-architecture}
\end{figure}

The network domain is the domain of the alleged new owner of the device, i.e the network that is expecting the device to be connected to it. The domain is a network of entities who share a common local trust anchor. 
It incorporates a \gls{pki} to govern the issuance of digital certificates to provide unique digital identifiers for clients and establish end-to-end security. 
\par
A join proxy is another component of the domain. It helps with the discovery of pledges that intend to join the network. In addition, it is responsible for discovering the domain registrar(s) and determining the proxy mechanisms supported by the registrar and utilizing the lowest impact mechanism.
Pledge discovery methods can be classified into passive and active methods. 
GRASP flooding \cite{rfc8990} is a pledge passive discovery method for autonomic networks \cite{kephart2003vision}. It is short for GeneRic Autonomic Signaling Protocol. It is used for signaling between autonomic service agents. GRASP provides discovery, flooding, synchronization, and negotiation functionalities for the technical objectives through respective GRASP messages. Pledge discovery via GRASP multicast flooding is the normative and mandatory method for \gls{brski}.
On the other hand, DNS-based Service Discovery \cite{rfc6763} over Multicast DNS \cite{rfc6762} as well as DHCP \cite{rfc2131} are pledge active discovery methods. They can be used as secondary discovery methods in parallel to GRASP.
Moreover, A proxy provides HTTPS connectivity and forwards messages without examination between a pledge and a registrar in the network, and without interfering with the protocol messages.
\par
A pledge is an unconfigured device attempting to join the network domain. Its goal is to be securely bootstrapped in a zero-touch fashion. To achieve this goal, the pledge establishes a TLS connection with one or more of the domain's registrars through the domain's proxy. It is necessary for the pledge and the registrar to establish mutual authentication. A manufacturer installed \gls{idevid} is used for pledge authentication to the domain's registrar. It is installed during the manufacturing process and includes certificates signed by the manufacturer and unique identifiers that represent the pledge, in addition to, the pledge unique serial number given by the manufacturer. It is recommended that The provided certificates are used for authentication with the registrar and the signing of voucher requests. The unique serial number is used in vouchers and voucher requests to ensure linkability. 
\par
A registrar is an element of the domain that is responsible to carry out the bootstrap process for the pledge. Also, it can be considered as the \gls{ra} for the domain's \gls{pki}. A domain can have one or more registrars which all have to be recognized by the domain proxy. If the pledge is capable to concurrently connect to multiple registrars, it is advisable to do so as this protects against a malicious proxy attempting a \gls{dos} attack like Slowloris.
\par
A manufacturer is the entity that produced the device and set up its initial configuration (IDevID). It provides two distinct services: the \gls{masa} service and ownership tracking and validation.
MASA can be a third party service that signs the vouchers issued for the bootstrapping process. It is also responsible for providing a repository for audit-log information of bootstrapping events. The service is contacted each time a pledge performs a zero-touch bootstrap in an attempt to enroll into a domain. It takes the decision whether or not issue the voucher according to the MASA policy. Voucher issuance could be done blindly at the lowest security level or it could be tightly bound to the sales channel that verifies the actual ownership of the domain. Hence, the manufacturer can provide protection against stolen devices or illegitimate resale of devices by declining voucher issuance to the suspected pledge.

Ownership tracking and validation is an optional manufacturer service. It is supposed to log all claim attempts and to know which device is owned by which domain and provide such information to registrars.  A verified log entry indicates that the pledge was issued a voucher as a result of positive verification of ownership.

\subsection{Protocol Details}
This section describes the message sequence of BRSKI illustrated in figure \ref{brski-protocol} and elaborates on content of the exchanged messages. The numbering sequence referenced through this section refers to the message numbers in figure \ref{brski-protocol}.

\begin{figure}[htbp]
	\centering
	\includegraphics[scale=0.4]{Images/brski-architecture.png}
	\caption{A successful BRSKI protocol run.}
	\label{brski-protocol}
\end{figure}

\begin{itemize}
	\item \textit{Message 1a, 1b (Discovery phase):} It is the first phase of the protocol where the pledge identifies the domain proxy. This can be performed by a pledge initiated mechanism as in message (1a) or via a proxy initiated mechanism as in message (1b). After successful discovery, the pledge can address a domain registrar through the proxy. A proxy does not assume any specific TLS version.

	\item \textit{Message 2, 3 (Provisional TLS establishment):} The pledge attempts to establish a TLS channel with each discovered registrar to ensure End-to-End security. The pledge must not use any TLS version lower than TLS 1.2, while TLS 1.3 is the encouraged version to be used. To establish the channel, mutual authentication has to be performed.
	At first, in message (2), the pledge receives the registrar's server certificate. However, the pledge does not posses any trust anchors to verify it yet. Therefore, the pledge accepts the registrar's certificate provisionally.
	Next in message (3), the pledge is authenticated via the installed IDevID. The registrar must be able to verify the provided certificate, however the distribution of the trust anchors for this task is out-of-scope of BRSKI.
	Meaning, the information received by the pledge must be untrusted, although it is in a TLS channel, till a trusted trust anchor to verify the certificate is received. 

	\item \textit{Message 4:} Having established a secure provisional TLS channel, the pledge initiates the voucher exchange by sending a pledge voucher request to the registrar in message (4). The request must contain a unique nonce per bootstrapping attempt to protect against replay attacks. Also, the request contains the `proximity-registrar-cert' and the pledge serial number.
	The `proximity-registrar-cert' is the \gls{ee} certificate of the registrar which is also used to establish the provisional TLS channel. Pledge serial number is a manufacturer defined unique identifier for each device. It is different from the IDevID certificate serial number. Since not all devices have a real-time clock, depending on the device capabilities, the request is recommended to have the 'created-on' value. Finally, the request must be signed using the pledge's IDevID certificate.
	\par
	The registrar authorizes the pledge based on the authenticated information presented in the pledge's IDevID and the registrar's policy. The policy can be either to allow any device from a specific vendor, to allow any device of a specific type, or to allow a specific type of devices from a specific vendor.
	
	\item \textit{BRSKI-MASA TLS channel:} The registrar initiates a TLS 1.2 or newer channel with the MASA where all subsequent communication between the two parties occur within this secure channel. The MASA URL is obtained from the pledge IDevID, as mentioned earlier. To authenticate the MASA, the registrar should be configurable with trust anchors on a per vendor MASA basis as part of the sales process. Moreover, the registrar should also support client authentication mechanisms such as TLS client certificate, HTTP Basic, Digest, or \gls{scram}; however TLS Client Certificate based authentication is the recommended method.
	
	\item \textit{Message 5:} After obtaining the pledge's voucher request, the registrar constructs a registrar voucher request that is sent to the MASA to obtain a voucher for the pledge. The registrar voucher request is a JSON document that is signed using a CMS structure. The JSON document encapsulates the pledge voucher request CMS object that was sent to the registrar and is referred to as `prior-signed-voucher-request'. Moreover, the request contains the `created-on' field which holds the timestamp the request was formed on. In addition, it consists of other fields which relate to the pledge request like the pledge serial number, the nonce used produced by the pledge and used in the pledge request, and `idevid-issuer' field which holds the issuer value of the pledge IDevID certificate. The registrar includes some certificates in the registrar voucher request CMS object as well. Those certificates are used by the MASA to be pinned into the voucher to be later used by the pledge as a trust anchor for authenticating the domain registrar. Therefore, the certificates enclosed by the registrar in the request have to be part of the chain it wishes the MASA to pin in the voucher. Hence the specificity of the attached certificates is considerably significant. A `pinned-domain-cert' can be as specific as the registrar's TLS \gls{ee} certificate. On the other hand, if it is as general as a public webPKI \gls{ca} it could permit any entity that possess a certificate issued by that authority to claim ownership of the device.
	\par
	On the other hand, the pledge might not be available at the time of deployment to send a pledge voucher request, or the registrar speculates to not being able to reach the MASA at the time of deployment where the pledge will be available. Such use cases justify the need for nonceless registrar voucher request. In these cases, the previous message (4) would not exist. To formulate this request, the registrar has to acquire the pledge's serial number and IDevID issuer, however, they are obtained through out-of-band means. Subsequently, the nonce field of the request is omitted.
	
	\item \textit{Phase 6 (Polling)}: Before processing the pledge's request, the registrar may send the pledge an HTTP 202 response message which indicates that the request received earlier has been accepted for processing however processing is not yet complete. This response initiates a polling phase between the pledge and the registrar. A ``Retry-After" field is specified within the headers of the registrar's response that indicates the minimum time for the pledge to wait before asking for a response for the voucher request sent earlier. After the specified waiting time, the pledge polls the response by resending the exact same request and must not change the nonce nor sign a new voucher request. If the pledge is simultaneously trying to bootstrap itself with several registrars of the network, it can be overwhelming for the pledge to keep track of all the ``Retry-After" times. Therefore, a pledge may ignore the specified interval and follow a hard-coded ``Retry-After" interval. A pledge should be able to hold the retry state for a maximum of 4 days.
	
	\item \textit{Message 7:} upon receiving the voucher request, the MASA performs a set of checks to decide weather to issue the requested voucher. Given the fact that vouchers have a short lifetime, the request may be from a registrar that has been issued a voucher previously, i.e a voucher renewal request. In this case, the request should be automatically authorized by the MASA.
	\par
	 The MASA extracts the certificate chain attached in the signed CMS object. If the domain CA is unknown to the MASA it is considered as a temporary trust anchor as the intention is not to authenticate the message rather to establish consistency of the domain PKI. According to the MASA's policy, it decides which certificate of the chain supplied by the registrar it chooses to pin. It may be the farthest certificate of the chain, or it may be as close as the \gls{ee} TLS certificate of the registrar. If revocation information is available for that certificate, it must be checked by the MASA to prevent issuance of new or renewed vouchers to unauthorized registrars. Next, the CMS signature is validated using the domain's CA extracted from the voucher request. Also, the signing certificate is verified to contain the `id-kp-cmcRA' Extended Key Usage. This ensures that the signer is an entity that is authorized to be a registrar of the domain. Hence, assures domains that a MASA only accepts requests from domain registrars. 
	 \par
	 In case of nonceless requests, It is mandatory for the MASA to authenticate the registrar. The decision to issue a noncless voucher is taken according to the MASA policy that is out of scope.
	 \par 
	 In case of nonced voucher requests, the MASA verifies that the `prior-signed-voucher-request', enclosed in the registrar request, contains a `proximity-registrar-cert' that is coherent to the certificate used to sign the registrar voucher request. Moreover, the nonce is verified to be consistent between the registrar voucher request and the `prior-signed-voucher-request'.
	 \par
	 Subsequent to a successful validation of the request, the MASA responds with an issued voucher in message (6). Any issued voucher by the MASA is recorded in the audit-log. Otherwise if a problem occurs, a response with the appropriate http signaling as described in \cite{brski}. For example, a 403 status code response if the voucher request is not signed correctly, or a 406 status code response if the requested voucher type or algorithms cannot be issued due to the MASA's awareness that such pledge is not capable of processing them.
	 
	\item \textit{Message 8:} The registrar evaluates the received voucher solely for transparency and future audit-log verification. The received voucher is forwarded to the pledge without any interference or modification from the registrar.

	\item \textit{Message 9:}
 	 After the pledge successfully receives a voucher, the pledge must indicate its status regarding the voucher to the domain. This occurs by sending a status message to the registrar. The pledge decides weather to accept the voucher or not through the voucher validation process. If acceptable, the message should contain the version of BRSKI and a boolean status field to indicate the acceptance status. In case of an unacceptable voucher or a failure, the pledge is expected to fail gracefully. The message should contain a Reason field with a string commenting on the cause. Nevertheless, the Reason should not be excessively descriptive as it may be sent to an unauthenticated and potentially malicious registrar.
	 \par
	 Bearing the voucher, the pledge verifies its validity. It verifies the signature using the manufacturer installed MASA trust anchor. It verifies also that the serial number enclosed in the voucher matches its own. For nonced vouchers, the pledge verifies the voucher nonce corresponds to the nonce it sent earlier in the voucher request. However nonceless vouchers can be accepted according to pledge local policy. The pledge can be configured to always accept nonceless vouchers to realize the use case where the MASA is unreachable at the time of pledge deployment.
	 \par
 	 A pledge could be operating in other similar security reduced mode that skip voucher validation in favor of offline or emergency touch-based deployment bootstrapping procedures. For example, \gls{tofu} or physical presence methods such as the use of serial console or depressing a physical button during bootstrapping. However, \gls{tofu} must not be available unless a hardware-assisted \gls{nea} is supported. Meanwhile, it is only recommended for other methods of skipping voucher validation. This recommendation serves as a prevention against unintended use of offline methods when autonomic methods fail or are unavailable.
	 \par
	 Upon successful verification of the voucher, the voucher's pinned-domain-cert should be considered by the pledge as a trust anchor. The current provisional TLS connection between the pledge and the registrar is evaluated using the obtained trust anchor. The pledge verifies the registrar's TLS server certificate using the trust anchor's public key. If the registrar's credentials could be verified, either by directly matching the server certificate or through verifying a higher certificate in its chain, the pledge trust the TLS connection and it is not considered provisional any further.
	 
	\item \textit{Message 10:} After receiving the pledge status telemetry message, the registrar requests the MASA audit-log from the MASA. The log data helps the registrar make a knowledgeable decision regarding further proceeding of the bootstrapping process. The decision making criteria is based upon the security requirements of the registrar domain. Hence, the criteria is out of the protocol's scope.
	The request content is the exact same registrar voucher-request sent earlier to the MASA, but is directed through a different URI specific for requesting the audit log, which is ``/.well-known/brski/requestauditlog". Reusing the same message minimizes the required cryptographic and message operations on both ends. The registrar may reuse the cached voucher request and the MASA may take advantage of its internal state to correlate the message with the already verified request averting additional operations.
	
	\item \textit{Message 11:} 
	A MASA can infer the proper pledge log to be prepared from the ``idevid-issuer" and the ``serial-number" information included in the received request of the previous message. Instead of immediately responding with the audit-log, the MASA can a HTTP 201 ``Created" response with a URL in the ``Location" header field redirecting to actual audit-log. The response log is a JSON format document consisting of all the log entries associated with the pledge. Nevertheless, a MASA that sends out URLS has to ensure they are unpredictable to avoid enumeration attacks against device audit-logs.
	\par
	The log format structure consists of several entries: ``version", ``events", and ``truncation". ``version" is an integer value representing the log format version. ``events" is an array of event objects that are associated to the device. 
	Each of the event objects is comprised of a set of entries. The ``date" entry represents the event's timestamp in the format according to \cite{rfc3339}. 
	The ``domainID" is a unique identifier for the domain's registrar that encodes the pinned-domain-cert's SubjectKeyIdentifier or SPKI fingerprint in base64. A ``nonce", if exists, is a base64 encoding of the same nonce used in the voucher request and issued voucher. If it is a nonceless voucher, then the field should preferably be set to null rather than omitting it. 
	The ``assertion" field indicates the level of verification with which the MASA issued the voucher. It can have one of three values: ``verified", ``logged", and ``proximity"; the latter being the one supported by this protocol.
	``truncated" field shows the number of event truncations for the specified domainID.
	Lastly, since audit-logs can be arbitrarily large, duplicated or old entries may be truncated as an optimization for the log structure. The ``truncation" entry contains meta-information about truncated entries such as ``nonced duplicates", ``nonced duplicates", and ``arbitrary".
	\par
	On the registrar side, the received audit-log is vetted for discrepancies and unexpected behavior like the pledge previously imprinting to an unexpected domain or whether a certain domain possesses a nonceless voucher and can reset the device anytime. If the registrar's audit-log verification is successful, then the bootstrapping process is complete.
	
	\item \textit{Message 12:} 
	At this point, the pledge has a trust anchor allowing it to verify the registrar, as well as a trusted TLS channel between them. Therefore the environment is suitable to start a certificate enrollment protocol after which the pledge obtains digital certificates that authenticate it to the domain and authorize it to utilize the relevant domain services.
	\par
	BRSKI is described as an extension for \gls{est} that provides automated proposal instead of the originally manual authentication method that relies on the intervention of a human user. Hence it is recommended for a pledge to use EST following BRSKI as a certificate enrollment protocol as it is considered a harmonious integration.
	\par 
	Nonetheless, the succeeding certificate enrollment protocol is not limited to \gls{est}, however, a variety of certificate enrollment protocols can be used. Using \gls{est} is an example of a pull model where the EST server is the protocol initiating party. As an example of the push model architecture, \gls{cmp} can be used as the certificate enrollment protocol since the \gls{ee} is the initiating party of the protocol. 
	

\end{itemize}











%sztp section
\section{\acrfull*{sztp}}
\gls{sztp} is as well an outcome of the \gls{ietf} group. 
It introduces a bootstrapping technique for devices in initial factory state.
It aims at securely supplying networking devices with bootstrapping data without further actions required other than physical placement and connecting the power and network cables.
\gls{sztp} is primarily concerned with physical devices, however, it has the potential to be extended to logical entities like virtual machines. 
The protocol is not limited to provisioning trust anchors to the device but rather can update the device's boot, commit an initial configuration, and execute arbitrary scripts to handle auxiliary requirements.

\subsection{Architecture Overview}
The protocol's architecture is divided into three abstract entities: the owner, the manufacturer, and the device. Figure \ref{sztp-architecture} illustrates the protocol's architecture.
\begin{figure}[H]
	\centering
	\includegraphics[scale=0.4]{Images/sztp-overview.png}
	\caption{SZTP Architecture.}
	\label{sztp-architecture}
\end{figure}
The owner is used to refer to the person or organization that owns the device. Throughout the protocol, the owner term abstractly represents the owner network domain and its sub-entities without explicitly referring to them. An example of the owner's sub-entities are the domain CA and the domain registrar. An owner possesses an \textit{Owner Certificate} which is an X.509 certificate that identifies the owner's identity and binds it to its public key. In addition to validating the owner's identity, the certificate is used to by other entities to validate the owner's digital signature over received artifacts. A \gls{nms} is assumed to be part of every owner's structure. A \gls{nms} is a software used by network administrators to monitor software and hardware nodes in a network and logs data from those nodes for reporting. The bootstrapping process introduces newly admitted devices to the \gls{nms}. 
\par
The manufacturer is the entity that produced the device. The term is also used to refer to entities that the manufacturer delegates functions to. A \gls{masa} is an example of a third party manufacturer delegate entity. Each manufacturer is supposed to operate a \gls{masa} or most commonly delegate its functionality to a third party. The \gls{masa} is responsible for generating the voucher required by the device to authenticate the owner and bootstrap the device. Each of the manufacturer services and its related entities has its own certificate for authentication and signing. The manufacturer preserves a list of trusted voucher-signing authorities and well-known bootstrap servers. The shipped devices from the manufacturer are pre-configured with the list of trust anchors along with their own \gls{idevid} credentials. Owners are assumed to trust the same pre-configured trust anchors as the ones maintained by the manufacturer.
\par
The protocol defines three artifacts that are exchanged through out the protocol: Conveyed information, ownership voucher, and owner certificate. Those artifacts provide the device with all the required information needed for bootstrapping. The conveyed information is divided between redirect information and onboarding information.
Bootstrap servers are \gls{rstcnf} servers that provide bootstrapping artifacts to devices. Depending on the type of information a server provides, they could split into two types of servers: Redirect servers and Onboarding servers. Redirect servers only returns redirect information to clients while onboarding servers only return onboarding information. However, servers are not limited to mentioned categories only. There may exist a server that can provide both types on conveyed information.
\par
Redirect information redirects a device to another bootstrapping server. 
Redirect information encodes a list of bootstrap servers hostnames and an optional trust anchor certificate that the device can use to authenticate each bootstrap server with. Depending on the source, redirect information may be trusted or untrusted. It is trusted whenever obtained via a secure connection to a trusted bootstrap server or whenever it is signed by the device’s owners. Trusted redirect information is useful for enabling a device to establish a secure connection to a specified bootstrap server. Untrusted redirect information is useful for directing a device to a bootstrap server where signed data has been staged for it to obtain. Redirection acts as a guide for device to discover the bootstrapping servers capable of providing onboarding information.
\par
Onboarding information is a bundle of data required for a device to perform the bootstrapping process into the owner’s network. It includes information about the boot image a device is required to be running, an initial configuration a device must apply, and scripts to address arbitrary needs which the device must successfully execute. Onboarding information must be obtained from a trusted source, either through a secure connection to a trusted bootstrap source or the information obtained is signed by the device’s owner.
\par
Devices can obtain the needed bootstrapping data from various sources according to the owner's deployed infrastructure. However, The concern is focused on sources that can supply devices with the information in an automated manner. Here are examples of touchless bootstrapping sources.
\begin{itemize}

\item
A DNS server can be utilized as a form of a bootstrapping server. It is an attractive method for environments which already employ a DNS infrastructure. It does necessarily require communication with an internet-accessible third party DNS service. Using DNS as bootstrapping server is considered a touchless bootstrapping as it does not involve any external interaction. However, a DNS server is not a trusted source of bootstrapping information, even if DNSSEC \cite{dnssec} is used to authenticate the DNS records. The reason being that the device cannot verify if the returned domain belongs to its rightful owner. Therefore, the returned DNS records must either be signed for later verification by the device or the records must be processed provisionally.
\par
Devices supporting DNS server as a bootstrap server are offered two prioritized types of queries to the server: device-specific and device independent queries. Queries must first be attempted using multicast DNS before unicast DNS. A DNS response for the device-specific query can encode the three artifacts into the TXT records but the response must be signed. However, by DNS conventions, the size of signed data is large. Therefore, it is implausible for the signed onboarding information to fit the UDP-based DNS packet size. Thus, if signed onboarding is to be sent over DNS, it is expected to be transported over \gls{tcp} so as to handle the DNS TXT record size. Otherwise, only redirect information shall be returned. On the other hand, DNS response for device-independent queries does not support returning onboarding information because the DNS server is incapable of returning signed data in response to the device-independent query. Accordingly, only redirect information shall be returned.
\item
Similar to DNS servers, a DHCP server is another untrusted bootstrapping data source. It can be leveraged by deployments that already employ a DHCP infrastructure. However, a DHCP server is a limited bootstrapping source due to its lack of ability to transmit enough data to hold signed bootstrapping data. Nevertheless, a DHCP server can return to a device unsigned redirect information.
\item
Lastly, a specific bootstrap server can be deployed to provide bootstrapping data and receive data from devices. It is defined as a \gls{rstcnf} server that implements the YANG module described in \cite[Section~7]{sztp}. Moreover, it may be using \gls{tls} which may eliminate the requirement to sign the transmitted bootstrapping data if the bootstrap server is trusted. Otherwise, if the bootstrap server is not trusted by the device, the conveyed information must be signed or processed provisionally. Whether a bootstrap server is trusted or not depends on the device’s knowledge of the server’s trust anchor. A bootstrap server exposes two endpoints to communicate with devices. Namely, the ``get-bootstrapping-data” for devices to request bootstrapping data and the ``report-progress” for devices to report their bootstrapping process status back to the bootstrap server, such as warnings, errors, and the result of the process. A bootstrap server may be hosted by the owner or is an internet accessible server hosted by the manufacturer.


\end{itemize}


\subsection{Protocol Details}
% Pre sztp phase
SZTP is focused on the enrollment process between the device and its owner.
However, before initiating SZTP, the owner and the manufacturer must complete a pre-protocol phase where the owner orders the devices and enrolls itself to the manufacturer where they exchange vital information required for the owner to be able to run SZTP.
This phase can be referred to as the Device Ordering and Owner Enrollment phase.
At first, the owner orders the needed devices from the manufacturer. In turn, the manufacturer always provides the owner with the trust anchor it will need to verify the IDevID certificates the devices use.
If the manufacturer offers an internet-based bootstrap server, the manufacturer may also provide the owner with the necessary credentials to access the hosted bootstrap server to configure it.
Consequently, the owner configures its \gls{nms} with the information obtained from manufacturer. If a remote bootstrap server credentials were obtained, the \gls{nms} configures an owner trust anchor certificate onto the bootstrap server that is to be used as the `pinned-domain-cert' of the voucher at the time of dynamically generating the enrollment voucher. 
After the manufacturer concludes the order and the devices are shipped, it may send the owner the serial numbers of the devices, other meta information, or even nonceless ownership vouchers. The owner may configure its \gls{nms} with the received information to prepare its network infrastructure to enroll the expected devices.
\par
% SZTP phase
Eventually, the owner is in possession of the devices and physically sets them up in its environment. As soon as the unconfigured device is booted for the first time, it checks if its factory default configuration enables SZTP bootstrapping. If enabled, the device initiates the SZTP protocol and goes through a series of phases to bootstrap into the owner network infrastructure. Figure \ref{fig:sztp-protocol} mentions the said phases which are discussed below.

\begin{figure}[htbp]
	\centering
	\includegraphics[width=0.5\linewidth]{Images/sztp-Protocol}
	\caption{SZTP protocol phases.}
	\label{fig:sztp-protocol}
\end{figure}
\begin{enumerate}
	\item \textit{Discovery:} First of all, the device discovers and enumerates all the available sources of bootstrapping information. A legitimate discoverable source of bootstrapping data can be one of those discussed in section \ref{subsec:sztp-overview}. For each source supported by the device, the device attempts to obtain bootstrapping data from it. The Bootstrap sources approached in order of their close proximity to the device. A device only process one bootstrapping server at a time.
	
	\item \textit{Authentication:} Secondly, when contacting a bootstrap server, it is mandatory for the device to authenticate itself to the server. The device authenticates itself using its TLS client certificate which is the same as its IDevID certificate. The server validates the client TLS certificate through the manufacturer trust anchor obtain earlier. Moreover, the device may possess more trust anchor certificates other than its TLS client certificate. If the device supports connecting to remote well-known servers, then it must authenticate the server through a pre-configured list of trust anchors for well-known servers. In addition, the device requires trust anchor certificates from the manufacturer for validating the ownership voucher. Devices should have the certificates chain of trust anchor certificates up to and including the self-signed root certificate. 
	
	\item \textit{Bootstrap data transmission \& processing:} After establishing a connection to the bootstrap server, whether trusted or not, the device starts receiving artifacts to bootstrap itself. The type data acceptable from bootstrapping sources depends on their trust state from the devices point of view. Untrusted sources can provide signed or unsigned redirect information, bearing in mind that the server which the device was redirected to must provide the device with further redirect information. Contradictory to redirect information, untrusted sources can only provide signed onboarding information. On the other hand, trusted sources can provide both redirect and onboarding information regardless signed or not. Since trusted sources already establish an authentic and secure channel with devices, it is redundant to have the transmitted information signed, therefore unnecessary. Naturally, the ownership voucher and the owner certificate are signed in any case and relevant revocation information may be additionally included in case it cannot be obtained dynamically. If the device received bootstrapping information that does not obey the restrictions respective to its source's trust status, the device should discard the information and quit the bootstrapping process with the said source.
	\par
	Whenever a device receives signed data, it must validate the data signature before processing it regardless of the data type. Signed data must always be accompanied by an owner voucher and an owner certificate. To validate the signed data the device must first validate the owner voucher and certificate. The device begins by validating the voucher's signature using its pre-configured trust anchors. It also verifies that the voucher's creation timestamp was in the past if the device is equipped with an accurate clock, and that the voucher ``assertion" value is acceptable by the device. In addition, the serial number of the device has to match that of the voucher as well as the ``idevid-issuer", if present. If valid, the device extracts the now trusted ``pinned-domain-cert" certificate from the voucher. Next, the device authenticates the owner certificate by verifying its certificate path to the extracted ``pinned-domain-cert", and if specified in the voucher, the revocation status of the certificate chain used to sign the owner certificate has to be checked. Finally, if the owner certificate is validated and the conveyed information was truly signed by that certificate, only then can the signed redirect or onboarding information be processed.
	\par
	Redirect information provides a device with a list of bootstrapping servers. The device proceeds through the list till it reaches a bootstrap server it can bootstrap from. If a server provides the device with bootstrapping data, the device must attempt to process it before proceeding to the next bootstrap server. A bootstrap server may further redirect the device to other bootstrap servers. In this case, implementations must limit the number of recursive redirection to a maximum of ten redirects to avoid indefinite redirections. Trusted redirect information may be accompanied by a trust anchor certificate. Hence, the device must authenticate the redirected to server certificate against the certificate obtained from the preceding bootstrap server. If no trust anchor is provided or the redirect information is untrusted, the device must process the information provisionally.
	\par
	Onboarding information is parsed by the device to extract the required instructions to be executed to evidently bootstrap the device. Onboarding information defines the boot image a device must run, pre-configuration scripts, initial configuration to be committed, and post-configuration scripts. The device must process the received information in that order. The boot image is verified if it complies with the required boot image. If the requirement is not satisfied, the device must download the boot image from the URIs defined in the onboarding information. After verifying the installed image, the device must reboot which will lead it to restart the bootstrapping process but with the new boot image, but then the device will not halt at this step and will proceed to further steps. If pre-configuration scripts are specified, the device must execute them and log their outputs. If an initial configuration is specified, the device must apply it according to the approach stated in the ``configuration-handling" node. Similar to pre-configuration scripts, post-configuration scripts must be executed and their output must be logged.
	\par
	Whether a device should return progress reports of the bootstrapping process is decided according to the trust state of the bootstrapping server. A device must not send progress reports to untrusted sources. However if the source is trusted, the source devices the minimum verbosity level it requires from a device in the ``reporting-level" node of the ``get-bootstrapping-data" RPC-reply. Moreover, the device may send more reports than originally specified by the trusted source. If an error occurs during any of the stages, devices must roll back the current step and previous steps and exit the bootstrapping process. However, results of some steps of the bootstrapping process may be retained, like an updated boot image. When eligible, devices must send error reports whenever they occur at any of the onboarding information processing stages. The bootstrapping process with current source should be quitted and the device should start the process over with the next bootstrapping source, if available.
	
\end{enumerate}

\section{Protocols Comparison}
\gls{brski} and \gls{sztp} are two different protocols which tackle the same problem. Their main goal is to automatically and securely provision trust anchors to devices without any manual interference at the device. Hence, allowing such unconfigured devices to authenticate the identity of their new owners and establish secure channels for further processes like certificate enrollment.
This section compares and contrasts both protocols. In addition, Appendix \ref{appendix-A} provides a comparison between the two protocols and their terminology in a tabular form.
\par
%Architecural
As mentioned, \gls{brski} and \gls{sztp} share a similar aim. \gls{brski} aims at storing a root certificate on the pledge that is sufficient for verifying the owner's domain registrar identity. While \gls{sztp} is more comprehensive as it involves updating the pledge boot image, commit an initial configuration, and execute arbitrary scripts.
The fundamental architecture for both protocols is the same as both function on three elemental entities: the pledge, an owner representative known as registrar, and the manufacturer's MASA. However, unlike \gls{brski}, the communication between the owner's registrar and the MASA is not intrinsic to the protocol. \gls{brski} is integrated with \gls{est} as an enrollment protocol to be utilized after bootstrapping, but it is not limited to it. On the other hand, \gls{sztp} is not natively integrated with an enrollment protocol. Both protocols use HTTP on top of TLS as their transport protocol, but \gls{brski} specifically restricts the use of TLS to version 1.2 or higher. In addition, \gls{brski} proposes the usability of CoAP as an alternative for HTTP. Both protocols do not require any specific cryptographic algorithm. 
\par
%bootstrapping data
The bootstrapping data differ between both protocols. For \gls{sztp}, the bootstrapping data are composed of the types of conveyed information explained earlier, the ownership voucher, and owner certificate. However, for \gls{brski}, it is only the ownership voucher artifact. Therefore, both protocols are reliant on the voucher artifact to provide the trust anchor necessary to verify the owner certificate. The owner requests the voucher similarly in both protocols. Nonceless vouchers are provided to owners out-of-band, while nonced vouchers are requested dynamically during a protocol run. In general, bootstrapping data are protected. In \gls{brski} and \gls{sztp}'s untrusted channels the data must be signed, but for \gls{sztp}'s trusted TLS channels the data may be optionally signed.
\par
% server
Owners supporting either protocol provide means to discover pledges and for pledges to obtain bootstrapping data from. For \gls{brski}, it depends on a domain proxy to discover pledges mDNS or GRASP broadcast messages. It introduces only one bootstrap source which the domain registrar. The registrar acts in principal as the owner's \gls{ca} and is responsible for bootstrapping pledges as well as communicating with the manufacturer services. Meanwhile, \gls{sztp} provides more options for pledges for discovery. An owner can administer a DNS server, a DHCP server, or a redirect server to respond to pledges discovery broadcast message and redirect the pledge to a source it can bootstrap from. However, similar to \gls{brski}, pledges can only obtain bootstrapping information from one source, namely, a bootstrap server. That is a server which is capable of providing redirect information as well as onboarding information. Nevertheless, bootstrapping sources of both protocols can be hosted locally at the owner side or remotely by a trusted third party.
\par
Both protocols allow owners to authorize the pledges which are allowed to initiate the protocol according to their vendor-specific serial numbers which are known in advance to a protocol run. 
Moreover, \gls{brski} can optionally couple the serial number with a specific pledge type and/or a specific vendor.
\par
An owner needs to know the URI for the MASA responsible for issuing vouchers for the pledge running the bootstrapping protocol. \gls{brski} recommends that the IDevID of the pledge contains URI for MASA to contact. On the other side, \gls{sztp} does not propose an inherent method to obtain the MASA URI. It rather relies on the \gls{oob} owner-manufacturer enrollment process to acquire the URI. After contacting the MASA, owners decide on whether to accept to bootstrap the pledge. In both protocols, the decision is made according to the verification of the attributes of the issued voucher. Also, \gls{brski} checks the MASA audit-log for further verification. As \gls{sztp} is more concerned with the pledge-owner communication, checking of the MASA audit-log is not discussed. However, \gls{sztp} does not limit it.
\par
\gls{brski} allows for pledge ownership transfer through issuance of new vouchers. Even though it is also feasible in \gls{sztp}, it is out of the protocol's scope. On the contrary, \gls{sztp} allows owners to provision domain specific configuration to pledges during the bootstrapping process, but \gls{brski} is not capable of providing such information.
\par
%device
A pledge is the protocol initiator for both protocols. Pledges compatible with either protocol must have an initial pre-configured state with a set of minimum required parameters. The initial state must contain the pledge IDevID and the trust anchors needed to verify the \gls{masa}. Optionally for both protocols, the device can be configured with a list of well-known remote bootstrap servers and their respective trust anchors. Moreover, \gls{sztp} pledges may be configured with a TLS client certificate and its related intermediate certificates. The initial state trust anchors are not updatable via either protocol methods.
\par
The bootstrap source must be authenticated even if it is not yet trusted. For either protocol, pledges establish a TLS channel with the bootstrap source. But if the pledge is not able to validate the source against one of its trust anchors, the TLS channel is provisionally trusted.
\par
For both protocols, pledges must check the validity of received vouchers with respect to their expiry time. This is dependent on whether the device has an accurate clock and on whether the voucher is nonced or nonceless, as expiry time might be omitted in nonced voucher. Pledges must also check the revocation status of received certificates, if revocation information is provided or if revocation checks are enforced.
\par
On the Manufacturer side, \gls{brski} optionally supports manufacturer tracking, while \gls{sztp} does not. Each protocol relies on the MASA for voucher issuance. However, \gls{brski} relies on the MASA for further functionalities it supports such as voucher renewal and maintaining and providing the voucher audit-log.
