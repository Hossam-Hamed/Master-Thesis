\subsection{Protocol Details}
% Pre sztp phase
SZTP is focused on the enrollment process between the device and its owner.
However, before initiating SZTP, the owner and the manufacturer must complete a pre-protocol phase where the owner orders the devices and enrolls itself to the manufacturer where they exchange vital information required for the owner to be able to run SZTP.
This phase can be referred to as the Device Ordering and Owner Enrollment phase.
At first, the owner orders the needed devices from the manufacturer. In turn, the manufacturer always provides the owner with the trust anchor it will need to verify the IDevID certificates the devices use.
If the manufacturer offers an internet-based bootstrap server, the manufacturer may also provide the owner with the necessary credentials to access the hosted bootstrap server to configure it.
Consequently, the owner configures its \gls{nms} with the information obtained from manufacturer. If a remote bootstrap server credentials were obtained, the \gls{nms} configures an owner trust anchor certificate onto the bootstrap server that is to be used as the `pinned-domain-cert' of the voucher at the time of dynamically generating the enrollment voucher. 
After the manufacturer concludes the order and the devices are shipped, it may send the owner the serial numbers of the devices, other meta information, or even nonceless ownership vouchers. The owner may configure its \gls{nms} with the received information to prepare its network infrastructure to enroll the expected devices.
\par
% SZTP phase
Eventually, the owner is in possession of the devices and physically sets them up in its environment. As soon as the unconfigured device is booted for the first time, it checks if its factory default configuration enables SZTP bootstrapping. If enabled, the device initiates the SZTP protocol and goes through a series of phases to bootstrap into the owner network infrastructure. Figure \ref{fig:sztp-protocol} mentions the said phases which are discussed below.

\begin{figure}[htbp]
	\centering
	\includegraphics[width=0.5\linewidth]{Images/sztp-Protocol}
	\caption{SZTP protocol phases.}
	\label{fig:sztp-protocol}
\end{figure}
\begin{enumerate}
	\item \textit{Discovery:} First of all, the device discovers and enumerates all the available sources of bootstrapping information. A legitimate discoverable source of bootstrapping data can be one of those discussed in section \ref{subsec:sztp-overview}. For each source supported by the device, the device attempts to obtain bootstrapping data from it. The Bootstrap sources approached in order of their close proximity to the device. A device only process one bootstrapping server at a time.
	
	\item \textit{Authentication:} Secondly, when contacting a bootstrap server, it is mandatory for the device to authenticate itself to the server. The device authenticates itself using its TLS client certificate which is the same as its IDevID certificate. The server validates the client TLS certificate through the manufacturer trust anchor obtain earlier. Moreover, the device may possess more trust anchor certificates other than its TLS client certificate. If the device supports connecting to remote well-known servers, then it must authenticate the server through a pre-configured list of trust anchors for well-known servers. In addition, the device requires trust anchor certificates from the manufacturer for validating the ownership voucher. Devices should have the certificates chain of trust anchor certificates up to and including the self-signed root certificate. 
	
	\item \textit{Bootstrap data transmission \& processing:} After establishing a connection to the bootstrap server, whether trusted or not, the device starts receiving artifacts to bootstrap itself. The type data acceptable from bootstrapping sources depends on their trust state from the devices point of view. Untrusted sources can provide signed or unsigned redirect information, bearing in mind that the server which the device was redirected to must provide the device with further redirect information. Contradictory to redirect information, untrusted sources can only provide signed onboarding information. On the other hand, trusted sources can provide both redirect and onboarding information regardless signed or not. Since trusted sources already establish an authentic and secure channel with devices, it is redundant to have the transmitted information signed, therefore unnecessary. Naturally, the ownership voucher and the owner certificate are signed in any case and relevant revocation information may be additionally included in case it cannot be obtained dynamically. If the device received bootstrapping information that does not obey the restrictions respective to its source's trust status, the device should discard the information and quit the bootstrapping process with the said source.
	\par
	Whenever a device receives signed data, it must validate the data signature before processing it regardless of the data type. Signed data must always be accompanied by an owner voucher and an owner certificate. To validate the signed data the device must first validate the owner voucher and certificate. The device begins by validating the voucher's signature using its pre-configured trust anchors. It also verifies that the voucher's creation timestamp was in the past if the device is equipped with an accurate clock, and that the voucher ``assertion" value is acceptable by the device. In addition, the serial number of the device has to match that of the voucher as well as the ``idevid-issuer", if present. If valid, the device extracts the now trusted ``pinned-domain-cert" certificate from the voucher. Next, the device authenticates the owner certificate by verifying its certificate path to the extracted ``pinned-domain-cert", and if specified in the voucher, the revocation status of the certificate chain used to sign the owner certificate has to be checked. Finally, if the owner certificate is validated and the conveyed information was truly signed by that certificate, only then can the signed redirect or onboarding information be processed.
	\par
	Redirect information provides a device with a list of bootstrapping servers. The device proceeds through the list till it reaches a bootstrap server it can bootstrap from. If a server provides the device with bootstrapping data, the device must attempt to process it before proceeding to the next bootstrap server. A bootstrap server may further redirect the device to other bootstrap servers. In this case, implementations must limit the number of recursive redirection to a maximum of ten redirects to avoid indefinite redirections. Trusted redirect information may be accompanied by a trust anchor certificate. Hence, the device must authenticate the redirected to server certificate against the certificate obtained from the preceding bootstrap server. If no trust anchor is provided or the redirect information is untrusted, the device must process the information provisionally.
	\par
	Onboarding information is parsed by the device to extract the required instructions to be executed to evidently bootstrap the device. Onboarding information defines the boot image a device must run, pre-configuration scripts, initial configuration to be committed, and post-configuration scripts. The device must process the received information in that order. The boot image is verified if it complies with the required boot image. If the requirement is not satisfied, the device must download the boot image from the URIs defined in the onboarding information. After verifying the installed image, the device must reboot which will lead it to restart the bootstrapping process but with the new boot image, but then the device will not halt at this step and will proceed to further steps. If pre-configuration scripts are specified, the device must execute them and log their outputs. If an initial configuration is specified, the device must apply it according to the approach stated in the ``configuration-handling" node. Similar to pre-configuration scripts, post-configuration scripts must be executed and their output must be logged.
	\par
	Whether a device should return progress reports of the bootstrapping process is decided according to the trust state of the bootstrapping server. A device must not send progress reports to untrusted sources. However if the source is trusted, the source devices the minimum verbosity level it requires from a device in the ``reporting-level" node of the ``get-bootstrapping-data" RPC-reply. Moreover, the device may send more reports than originally specified by the trusted source. If an error occurs during any of the stages, devices must roll back the current step and previous steps and exit the bootstrapping process. However, results of some steps of the bootstrapping process may be retained, like an updated boot image. When eligible, devices must send error reports whenever they occur at any of the onboarding information processing stages. The bootstrapping process with current source should be quitted and the device should start the process over with the next bootstrapping source, if available.
	
\end{enumerate}
