\section{Protocols Comparison}
\gls{brski} and \gls{sztp} are two different protocols which tackle the same problem. Their main goal is to automatically and securely provision trust anchors to devices without any manual interference at the device. Hence, allowing such unconfigured devices to authenticate the identity of their new owners and establish secure channels for further processes like certificate enrollment.
This section compares and contrasts both protocols. In addition, Appendix \ref{appendix-A} provides a comparison between the two protocols and their terminology in a tabular form.
\par
%Architecural
As mentioned, \gls{brski} and \gls{sztp} share a similar aim. \gls{brski} aims at storing a root certificate on the pledge that is sufficient for verifying the owner's domain registrar identity. While \gls{sztp} is more comprehensive as it involves updating the pledge boot image, commit an initial configuration, and execute arbitrary scripts.
The fundamental architecture for both protocols is the same as both function on three elemental entities: the pledge, an owner representative known as registrar, and the manufacturer's MASA. However, unlike \gls{brski}, the communication between the owner's registrar and the MASA is not intrinsic to the protocol. \gls{brski} is integrated with \gls{est} as an enrollment protocol to be utilized after bootstrapping, but it is not limited to it. On the other hand, \gls{sztp} is not natively integrated with an enrollment protocol. Both protocols use HTTP on top of TLS as their transport protocol, but \gls{brski} specifically restricts the use of TLS to version 1.2 or higher. In addition, \gls{brski} proposes the usability of CoAP as an alternative for HTTP. Both protocols do not require any specific cryptographic algorithm. 
\par
%bootstrapping data
The bootstrapping data differ between both protocols. For \gls{sztp}, the bootstrapping data are composed of the types of conveyed information explained earlier, the ownership voucher, and owner certificate. However, for \gls{brski}, it is only the ownership voucher artifact. Therefore, both protocols are reliant on the voucher artifact to provide the trust anchor necessary to verify the owner certificate. The owner requests the voucher similarly in both protocols. Nonceless vouchers are provided to owners out-of-band, while nonced vouchers are requested dynamically during a protocol run. In general, bootstrapping data are protected. In \gls{brski} and \gls{sztp}'s untrusted channels the data must be signed, but for \gls{sztp}'s trusted TLS channels the data may be optionally signed.
\par
% server
Owners supporting either protocol provide means to discover pledges and for pledges to obtain bootstrapping data from. For \gls{brski}, it depends on a domain proxy to discover pledges mDNS or GRASP broadcast messages. It introduces only one bootstrap source which the domain registrar. The registrar acts in principal as the owner's \gls{ca} and is responsible for bootstrapping pledges as well as communicating with the manufacturer services. Meanwhile, \gls{sztp} provides more options for pledges for discovery. An owner can administer a DNS server, a DHCP server, or a redirect server to respond to pledges discovery broadcast message and redirect the pledge to a source it can bootstrap from. However, similar to \gls{brski}, pledges can only obtain bootstrapping information from one source, namely, a bootstrap server. That is a server which is capable of providing redirect information as well as onboarding information. Nevertheless, bootstrapping sources of both protocols can be hosted locally at the owner side or remotely by a trusted third party.
\par
Both protocols allow owners to authorize the pledges which are allowed to initiate the protocol according to their vendor-specific serial numbers which are known in advance to a protocol run. 
Moreover, \gls{brski} can optionally couple the serial number with a specific pledge type and/or a specific vendor.
\par
An owner needs to know the URI for the MASA responsible for issuing vouchers for the pledge running the bootstrapping protocol. \gls{brski} recommends that the IDevID of the pledge contains URI for MASA to contact. On the other side, \gls{sztp} does not propose an inherent method to obtain the MASA URI. It rather relies on the \gls{oob} owner-manufacturer enrollment process to acquire the URI. After contacting the MASA, owners decide on whether to accept to bootstrap the pledge. In both protocols, the decision is made according to the verification of the attributes of the issued voucher. Also, \gls{brski} checks the MASA audit-log for further verification. As \gls{sztp} is more concerned with the pledge-owner communication, checking of the MASA audit-log is not discussed. However, \gls{sztp} does not limit it.
\par
\gls{brski} allows for pledge ownership transfer through issuance of new vouchers. Even though it is also feasible in \gls{sztp}, it is out of the protocol's scope. On the contrary, \gls{sztp} allows owners to provision domain specific configuration to pledges during the bootstrapping process, but \gls{brski} is not capable of providing such information.
\par
%device
A pledge is the protocol initiator for both protocols. Pledges compatible with either protocol must have an initial pre-configured state with a set of minimum required parameters. The initial state must contain the pledge IDevID and the trust anchors needed to verify the \gls{masa}. Optionally for both protocols, the device can be configured with a list of well-known remote bootstrap servers and their respective trust anchors. Moreover, \gls{sztp} pledges may be configured with a TLS client certificate and its related intermediate certificates. The initial state trust anchors are not updatable via either protocol methods.
\par
The bootstrap source must be authenticated even if it is not yet trusted. For either protocol, pledges establish a TLS channel with the bootstrap source. But if the pledge is not able to validate the source against one of its trust anchors, the TLS channel is provisionally trusted.
\par
For both protocols, pledges must check the validity of received vouchers with respect to their expiry time. This is dependent on whether the device has an accurate clock and on whether the voucher is nonced or nonceless, as expiry time might be omitted in nonced voucher. Pledges must also check the revocation status of received certificates, if revocation information is provided or if revocation checks are enforced.
\par
On the Manufacturer side, \gls{brski} optionally supports manufacturer tracking, while \gls{sztp} does not. Each protocol relies on the MASA for voucher issuance. However, \gls{brski} relies on the MASA for further functionalities it supports such as voucher renewal and maintaining and providing the voucher audit-log.