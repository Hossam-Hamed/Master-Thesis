% !TeX spellcheck = en_US
% !TeX encoding = UTF-8
\chapter*{Abstract}
\gls{iot} is being hailed as one of the primary catalysts of the next digital revolution. The technology enables connecting ubiquitous objects to the internet. 
Security is an essential aspect of the different phases of the lifecycle of an IoT device.
In particular, our thesis is primarily focused on discussing the security of the bootstrapping phase and management of cryptographic keys during the operational phase to ensure desired security features for the utilized keys.
The first goal of this thesis is to examine the significance of secure zero-touch bootstrapping of IoT devices. We analyze and present a comparison between two bootstrapping protocols: \gls{brski} and \gls{sztp}. 
Secondly, we aim to discuss the future secrecy of cryptographic keys in IoT through analyzing the \gls{x3dh} protocol and the double ratchet algorithm. In addition to use \gls{ofmc} to formally verify a model of a \gls{x3dh} protocol variant. Moreover, provide an implementation for the future secrecy related protocols.
Thirdly, we discuss the applicability of the mentioned protocols in IoT-related use cases.
Lastly, we aimed at conducting a comparative study between certificate enrollment protocols.
Although the thesis does not finally conclude the discussion of IoT security, we have shown than that the mentioned bootstrapping protocols
Regarding X3DH protocol and the double ratchet algorithm, they are viable solutions and applicable to IoT and provide and enhance security in the proposed scenarios.
