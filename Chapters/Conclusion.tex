\chapter{Conclusion}
\label{ch:conclusion}
The goal of this thesis is to examine and comprehend the significance of secure zero-touch bootstrapping of IoT devices, as well as the future secrecy of cryptographic keys in IoT protocols. In addition, to give an implementation for future secrecy-related protocols in relevant use cases.
At first, we presented in chapter \ref{ch:background} an overview of different concepts and cryptographic primitives, as well as related work, which are a foundation for understanding the work presented through out the thesis. Next in chapter \ref{ch:usecases}, Three use cases are introduced: \acrfull{abc}, Industrial IoT, and V2X communication. For each use case, we further discuss the relevance of secure communication and the challenges it faces. In chapter \ref{ch:secureBootstrapping}, we first give a general overview of secure bootstrapping. Afterwards, for each of the zero-touch protocols, \gls{brski} and \gls{sztp}, we give an overview of the protocol architecture, in addition to explaining the execution details for each protocol to accomplish bootstrapping of a pledge. Moreover, we compare and contrast both protocols. Furthermore, in chapter \ref{ch:postcomp}, we discussed the post-compromise security property, aka future secrecy, in the realm of secure messaging and how it differs from forward secrecy. Also, we explain two protocols, which are part of the signal protocol, that work together to achieve desirable secure messaging properties ---in the context of two party communication only--- that include future secrecy, \gls{x3dh} and the double ratchet. For the \gls{x3dh}, we additionally use \gls{ofmc} to perform a formal verification for a model of the protocol. The model is a modified version from the original specification to guarantee its security under the \gls{ofmc} intruder model. Nevertheless, our variant scarifies the deniability property offered by the original protocol. For both protocols, we discuss their post-quantum security and other worked related to their formal verification. In chapter \ref{ch:implementation}, we present a demo implementation of the protocols discussed in chapter \ref{ch:postcomp}. Chapter \ref{ch:discussion} discusses the usage of the explained protocols in chapters \ref{ch:secureBootstrapping} and \ref{ch:postcomp} in the use cases defined earlier. Finally, we present a comparison of certificate enrollment protocols in appendix \ref{appendix-enrollment}. 
