\chapter{Introduction}
\label{ch:introduction}

\gls{iot} is a relatively new concept that already has applications in many domains, creating new ways of interactions between humans and small devices, referred to as smart devices. Applications like Smart Airports, Transportation, Home Automation, and many more are being revolutionized by this technology \cite{marksteiner2017overview}.
\gls{iot} is the deployment of network-connected constrained devices that interact with the physical environment by employing sensors to collect data, managing other systems, controlling actuators, as well as communicating with one another. \gls{iot} devices are classified into several classes depending on their degree of computing resources and power usage. Depending on their degree, they may be assumed to not be able to process sophisticated or even conventional cryptographic operations. In many use cases, the \gls{iot} devices handle critical and private data. This imposes the requirement for data integrity and authenticity guarantees, and -because of privacy- in many cases also confidentiality.
\par
IoT device life cycle and nature of IoT communication 

%\gls{tls} \cite{rfc5246}, which relies on \gls{tcp}, has been criticized as inappropriate for constrained \gls{iot} devices \cite{shang2016challenges}. Among the reasons are, the infeasibility of \gls{iot} devices to maintain long-lived connections due to energy constraints, high header overhead, and the low-latency requirement that is opposed by the delay due to \gls{tcp} handshake, especially in a lossy network.
%Alternatively, \gls{dtls} \cite{dtls} provides security for communication channels relying on \gls{udp}. In contrast to \gls{tcp}, \gls{udp} is more suitable for \gls{iot}. It is an unreliable protocol as it does not care about message delivery , resulting in a lower header overhead. In addition, it introduces less traffic to the network.
%\par
%The \gls{ietf} introduced \gls{coap} \cite{rfc7252} that is intended to be a generic application protocol for constrained environments. Similar to the ubiquitous HTTP \cite{http}, \gls{coap} realizes a subset of the \gls{rest} architecture. Therefore, it easily translates to HTTP. Moreover, \gls{coap} leverages \gls{udp} as its transport layer protocol which is secured by \gls{dtls}. Hence, \gls{coap} is one of well-suited protocols for \gls{iot}.

\section{Research Questions and Contributions}\label{sec:reserach_questions}
research questions: analyze and understand the relevance and importance of zero-touch bootstrapping and future secrecy in IoT protocols and present a proposal for an implementation of future secrecy related protocols in relevant use cases.
\par
The contributions of the thesis are the following:
\begin{itemize}
	\item discuss applicability of protcols on use cases.
	\item We conduct a comparative study between two secure bootstrapping protocols, \gls{brski} and \gls{sztp}.
	\item We give an overview over the \gls{x3dh} protocol and the double ratchet algorithm. In addition to present a formal verification of the \gls{x3dh} using \gls{ofmc}. Moreover, we discuss the post-quantum security of the protocols.
	\item We provide a demo implementation for \gls{x3dh} and the double ratchet algorithm using Python.
	\item We present a comparison between a set of certificate enrollment protocols in appendix \ref{appendix-enrollment}.
\end{itemize}

\section{Structure of the Thesis}\label{sec:structure_of_the_thesis}

The structure of this thesis is as follows: In chapter \ref{ch:background}, we reflect on principles relevant to the understanding of the work presented in further chapters. In addition to other works related to our thesis. 
Next we propose real-world use cases in chapter \ref{ch:usecases} where the protocols to be proposed can be applied.
In chapter \ref{ch:secureBootstrapping}, we give an overview of the two promising automated bootstrapping protocols: \gls{brski} and \gls{sztp}. Moreover, we present a comparative study between them.
Chapter \ref{ch:postcomp} discusses post-compromise security, and how it is achievable in an asynchronous environment through the use of \gls{x3dh} and double ratcheting. In addition, we discuss the formal verification of the protocols and their stand in the world of quantum computing.
In chapter \ref{ch:implementation} we present our demo implementation of the protocols discussed in chapter \ref{ch:secureBootstrapping}. 
Furthermore, chapter \ref{ch:discussion} discusses our findings and reflects the represented protocols on the use cases.
Finally, chapter \ref{ch:conclusion} concludes our work and introduces suggestions for future work.